% Howto for Project Nonnatus
% Daniel Williams <dwilliams@port8080.net>

% Command for compiling:
% pdflatex nonnatus && pdflatex nonnatus

\documentclass[letterpaper]{article}

\title{Project Nonnatus \\ Nonnatus Server Howto}
\author{Daniel Williams \\ dwilliams@port8080.net}

\begin{document}
\maketitle

\section{Introduction}

\section{Hardware Recommendations}

\section{Installing the Software}

\subsection{OpenBSD}

% Fill this in based on a basic version from the openBSD website.

% Maybe add the user to the sudoers file (security issues?).

\subsection{Python}

To install python, either su - to the root account or the logged in user must
sudoer permissions.

% I'm going to install based on the python package for simplicity's sake.
\begin{enumerate}

\item
Export the \verb=PKG_PATH= variable to make life easy.
% The path for the Boulder, CO mirror: http://ftp3.usa.openbsd.org/pub/OpenBSD/5.4/packages/amd64/
\begin{verbatim}
export PKG_PATH=http://<mirror_url>/pub/OpenBSD/5.4/packages/<arch>/
\end{verbatim}

\item
Run \verb=pkg_add= to install python.  The sudo command will need to be
prepended to this command if using that method or privilege escalation.
\begin{verbatim}
pkg_add python-2.7.5
\end{verbatim}

\item
When the installation completes, make the symlinks for python.
\begin{verbatim}
ln -sf /usr/local/bin/python2.7 /usr/local/bin/python
ln -sf /usr/local/bin/python2.7-2to3 /usr/local/bin/2to3
ln -sf /usr/local/bin/python2.7-config /usr/local/bin/python-config
ln -sf /usr/local/bin/pydoc2.7  /usr/local/bin/pydoc
\end{verbatim}

\end{enumerate}

\subsection{Go}

To install go, either su - to the root account or the logged in user must
sudoer permissions.

% I'm going to install based on the python package for simplicity's sake.
\begin{enumerate}

\item
Export the \verb=PKG_PATH= variable to make life easy.
% The path for the Boulder, CO mirror: http://ftp3.usa.openbsd.org/pub/OpenBSD/5.4/packages/amd64/
\begin{verbatim}
export PKG_PATH=http://<mirror_url>/pub/OpenBSD/5.4/packages/<arch>/
\end{verbatim}

\item
Run \verb=pkg_add= to install go.  The sudo command will need to be prepended
to this command if using that method or privilege escalation.
\begin{verbatim}
pkg_add go
\end{verbatim}

\item
When the installation completes, check to make sure go is working correctly..
\begin{verbatim}
go version
\end{verbatim}

\end{enumerate}

\subsection{Least Authority File System}
% Grab the latest version.  Probably should have a "latest" link on the tahoe
% website.
% https://tahoe-lafs.org/source/tahoe-lafs/releases/allmydata-tahoe-1.10.0.zip

% I'm going to install this service as root for now.  It will most likely get
% a user specific to tahoe, or get wrapped up under the apache/www user.
To install tahoe-lafs, either su - to the root account or the logged in user must
sudoer permissions.

% I'm going to install based on the python package for simplicity's sake.
\begin{enumerate}

\item
Download the latest tarball from the tahoe-lafs website.  Make sure to grab a
\verb=.tar.bz2= or \verb=.tar.gz= file.
% The path for the Boulder, CO mirror: http://ftp3.usa.openbsd.org/pub/OpenBSD/5.4/packages/amd64/
% FIXME: This looks like ass as it overflows the right margin.  Find a better way to
% way to format this line:
\begin{verbatim}
lynx https://tahoe-lafs.org/source/tahoe-lafs/releases/allmydata-tahoe-1.10.0.tar.bz2
\end{verbatim}
Press ''d'' to download, then select ''Save to disk'' once the file has
downloaded, and press ''q'' then ''y'' to quit \verb=lynx=.

\item
Untarball the downloaded file.
\begin{verbatim}
tar xjf allmydata-tahoe-1.10.0.tar.bz2
\end{verbatim}

\item
Change to the freshly unpacked directory and run the build script.
\begin{verbatim}
cd allmydata-tahoe-1.10.0
python setup.py build
\end{verbatim}

\end{enumerate}

\subsection{Pond}

\section{Configuring the Software}

\subsection{Least Authority File System}

\subsection{Pond}

\section{Usage Recommendations}

\subsection{LAFS Basics}

\subsection{Pond Basics}

\subsection{Bad Habits}

\end{document}
